\documentclass[12pt,a4paper]{article}

\makeatletter
    \input{../config/header[fr].sty}

    \usepackage{01-demo-explained}
\makeatother


\begin{document}

\section{Université}

\vspace{22.5cm}

\begin{demoexplain}
    \demostep
        Hypothèse & $A$     
    \demostep
        Axiome 1  & $A \implies B$
    \demostep
        m.p. sur
        \explref*{1} et \explref*{2}
      & $B$
    \demostep
        \explref*{1} et \explref*{3}
      & $A \wedge B$
\end{demoexplain}



\setcounter{page}{0}
\newpage



\section{Collège et lycée}

\vspace{19cm}

\begin{demoexplain*}
    \demostep
        $ABC$ est un triangle \newline équilatéral 
      & Définition d'un triangle \newline équilatéral. 
      & $AB = BC = AC$
    \demostep{} % --> Ne pas oublier !
      & Voir l'énoncé.
      & $AB = 10 \, cm$
    \demostep
        Voir les conséquences \newline \explref*{1} et \explref*{2} .
      & Simple calcul.
      & $ABC$ a pour périmètre $30 \, cm$.
    %
    \demostep
        $ABC$ est un triangle \newline équilatéral 
      & Définition d'un triangle \newline équilatéral. 
      & $AB = BC = AC$
    \demostep{} % --> Ne pas oublier !
      & Voir l'énoncé.
      & $AB = 10 \, cm$
    \demostep
        Voir les conséquences \newline \explref*{4} et \explref*{5} .
      & Simple calcul.
      & $ABC$ a pour périmètre $30 \, cm$.
\end{demoexplain*}

\end{document}
