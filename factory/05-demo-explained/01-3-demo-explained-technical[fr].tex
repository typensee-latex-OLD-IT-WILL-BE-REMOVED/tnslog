\documentclass[12pt,a4paper]{article}

\makeatletter
    \usepackage[utf8]{inputenc}
\usepackage[T1]{fontenc}
\usepackage{ucs}

\usepackage[french]{babel,varioref}

\usepackage[top=2cm, bottom=2cm, left=1.5cm, right=1.5cm]{geometry}
\usepackage{enumitem}

\usepackage{pgffor}

\usepackage{multicol}

\usepackage{makecell}

\usepackage{color}
\usepackage{hyperref}
\hypersetup{
    colorlinks,
    citecolor=black,
    filecolor=black,
    linkcolor=black,
    urlcolor=black
}

\usepackage{amsthm}

\usepackage{tcolorbox}
\tcbuselibrary{listingsutf8}

\usepackage{ifplatform}

\usepackage{ifthen}

\usepackage{macroenvsign}


% Sections numbering

%\renewcommand\thechapter{\Alph{chapter}.}
\renewcommand\thesection{\Roman{section}.}
\renewcommand\thesubsection{\arabic{subsection}.}
\renewcommand\thesubsubsection{\roman{subsubsection}.}



% MISC

\newtcblisting{latexex}{%
	sharp corners,%
	left=1mm, right=1mm,%
	bottom=1mm, top=1mm,%
	colupper=red!75!blue,%
	listing side text
}

\newtcbinputlisting{\inputlatexex}[2][]{%
	listing file={#2},%
	sharp corners,%
	left=1mm, right=1mm,%
	bottom=1mm, top=1mm,%
	colupper=red!75!blue,%
	listing side text
}


\newtcblisting{latexex-flat}{%
	sharp corners,%
	left=1mm, right=1mm,%
	bottom=1mm, top=1mm,%
	colupper=red!75!blue,%
}

\newtcbinputlisting{\inputlatexexflat}[2][]{%
	listing file={#2},%
	sharp corners,%
	left=1mm, right=1mm,%
	bottom=1mm, top=1mm,%
	colupper=red!75!blue,%
}


\newtcblisting{latexex-alone}{%
	sharp corners,%
	left=1mm, right=1mm,%
	bottom=1mm, top=1mm,%
	colupper=red!75!blue,%
	listing only
}

\newtcbinputlisting{\inputlatexexalone}[2][]{%
	listing file={#2},%
	sharp corners,%
	left=1mm, right=1mm,%
	bottom=1mm, top=1mm,%
	colupper=red!75!blue,%
	listing only
}


\newcommand\inputlatexexcodeafter[1]{%
	\begin{center}
		\input{#1}
	\end{center}

	\vspace{-.5em}
	
	Le rendu précédent a été obtenu via le code suivant.
	
	\inputlatexexalone{#1}
}


\newcommand\inputlatexexcodebefore[1]{%
	\inputlatexexalone{#1}
	\vspace{-.75em}
	\begin{center}
		\textit{\footnotesize Rendu du code précédent}
		
		\medskip
		
		\input{#1}
	\end{center}
}


\newcommand\env[1]{\texttt{#1}}
\newcommand\macro[1]{\env{\textbackslash{}#1}}



\setlength{\parindent}{0cm}
\setlist{noitemsep}

\theoremstyle{definition}
\newtheorem*{remark}{Remarque}

\usepackage[raggedright]{titlesec}

\titleformat{\paragraph}[hang]{\normalfont\normalsize\bfseries}{\theparagraph}{1em}{}
\titlespacing*{\paragraph}{0pt}{3.25ex plus 1ex minus .2ex}{0.5em}


\newcommand\separation{
	\medskip
	\hfill\rule{0.5\textwidth}{0.75pt}\hfill
	\medskip
}


\newcommand\extraspace{
	\vspace{0.25em}
}


\newcommand\whyprefix[2]{%
	\textbf{\prefix{#1}}-#2%
}

\newcommand\mwhyprefix[2]{%
	\texttt{#1 = #1-#2}%
}

\newcommand\prefix[1]{%
	\texttt{#1}%
}


\newcommand\inenglish{\@ifstar{\@inenglish@star}{\@inenglish@no@star}}

\newcommand\@inenglish@star[1]{%
	\emph{\og #1 \fg}%
}

\newcommand\@inenglish@no@star[1]{%
	\@inenglish@star{#1} en anglais%
}


\newcommand\ascii{\texttt{ASCII}}


% Example
\newcounter{paraexample}[subsubsection]

\newcommand\@newexample@abstract[2]{%
	\paragraph{%
		#1%
		\if\relax\detokenize{#2}\relax\else {} -- #2\fi%
	}%
}



\newcommand\newparaexample{\@ifstar{\@newparaexample@star}{\@newparaexample@no@star}}

\newcommand\@newparaexample@no@star[1]{%
	\refstepcounter{paraexample}%
	\@newexample@abstract{Exemple \theparaexample}{#1}%
}

\newcommand\@newparaexample@star[1]{%
	\@newexample@abstract{Exemple}{#1}%
}


% Change log
\newcommand\topic{\@ifstar{\@topic@star}{\@topic@no@star}}

\newcommand\@topic@no@star[1]{%
	\textbf{\textsc{#1}.}%
}

\newcommand\@topic@star[1]{%
	\textbf{\textsc{#1} :}%
}



    \usepackage{01-demo-explained}
\makeatother


\newcommand\anglein[1]{#1}

\begin{document}

%\section{Détailler un raisonnement}

\subsection{Fiches techniques}

\paragraph{Détailler un \og vrai \fg{} raisonnement via un tableau}

\IDenv{demoexplain}{4}

\IDkey{start} le début de la numérotation des identifiants des justifications.
              La valeur par défaut est \verb+1+ et la valeur spéciale \verb+last+ permet de reprendre la numérotation là où elle s'était arrêtée le dernier environnement \env{demoexplain} ou \env{demoexplain*} utilisé.

\IDkey{hyps} les hypothèses, au format texte, vérifiées au départ.
             Cet argument peut être vide et ne doit pas rentrer en conflit avec l'option \verb+hyp+.

\IDkey{hyp} une unique hypothèse, au format texte, vérifiée au départ.
            Cet argument peut être vide et ne doit pas rentrer en conflit avec l'option \verb+hyps+.

\IDkey{ccl} la conclusion, au format texte, du raisonnement détaillé.
            Cet argument peut être vide.


\separation


\IDenv{demoexplain*}{1}

\IDkey{start} le début de la numérotation des identifiants des justifications.
              La valeur par défaut est \verb+1+ et la valeur spéciale \verb+last+ permet de reprendre la numérotation là où elle s'était arrêtée le dernier environnement \env{demoexplain} ou \env{demoexplain*} utilisé.


\separation


\IDmacro{demostep}{1}{0}

\IDoption{} un texte qui sera utilisé comme label global référençant le numéro d'une justification.


\separation


\IDmacro*{explref}{1}  où \quad \mwhyprefix{expl}{ain}
                             et \mwhyprefix{ref}{erence}

\IDarg{} un numéro de $1$ ou $2$ chiffres qui sera encadré comme le sont les numérotations des indications.


\separation


\IDmacro*{explref*}{1}  où \quad \mwhyprefix{expl}{ain}
                              et \mwhyprefix{ref}{erence}

\IDarg{} un texte correspondant à un label global référençant le numéro d'une justification.


\paragraph{Détailler un \og vrai \fg{} raisonnement via un tableau - Textes utilisés}

\IDmacro*{textdemoID}{0}     où \quad \mwhyprefix{ID}{entifier}

\IDmacro*{textdemoKNOWN}{0}

\IDmacro*{textdemoPROP}{0}   où \quad \mwhyprefix{PROP}{osition}

\IDmacro*{textdemoCONS}{0}   où \quad \mwhyprefix{CONS}{equence}

\extraspace

\IDmacro*{textdemoHYPS}{0}   où \quad \prefix{HYPS = HYP-othesis (S-everal ones)} 

\IDmacro*{textdemoHYP}{0}    où \quad \prefix{HYP = HYP-othesis (just one)}

\IDmacro*{textdemoCCL}{0}    où \quad \prefix{CCL = C-on-CL-usion}

\extraspace

\IDmacro*{textdemoNEXTPAGE}{0}


\end{document}
