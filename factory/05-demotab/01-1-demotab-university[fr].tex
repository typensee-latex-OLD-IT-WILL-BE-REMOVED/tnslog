\documentclass[12pt,a4paper]{article}

\makeatletter
    \input{../config/header[fr].sty}

    \usepackage{01-demotab}
\makeatother


\begin{document}

\section{Détailler un \og vrai \fg{} raisonnement}

\subsection{Un tableau pour le post-bac}

\newparaexample{Le minimum avec les réglages par défaut}

Prenons un exemple utile à la logique formelle en informatique théorique mais qui a complètement sa place en mathématiques plus classiques \emph{(voir la section \ref{tnslog-explain-hard-proof-for-youngs} pour un autre type de présentation plus adapté à un public de collège ou de lycée)}.
Ci-dessous l'environnement \env{demotab} facilite la mise en page
\footnote{
	En coulisse est utilisé l'environnement \env{longtable} du package éponyme.
}
et la macro étoilée \macro{explref*} permet d'indiquer une référence interne au raisonnement
\footnote{
    Les indications peuvent être numérotes jusqu'à $99$ ce qui est bien au-delà des besoins pratiques.
}.
Dans cet exemple en deux morceaux, pour montrer au passage comment continuer la numérotation là où elle s'était arrêtée, on utilise \emph{\og m.p. \fg} comme abréviation de \emph{\og modus ponens \fg}.

\begin{latexex}
\begin{demotab}
    \demostep
        Hypothèse & $A$     
    \demostep
        Axiome 1  & $A \implies B$
    \demostep
        m.p. sur
        \explref*{1} et \explref*{2}
                  & $B$
    \demostep
        \explref*{1} et \explref*{3}
                  & $A \wedge B$
                  
\end{demotab}
\end{latexex}


Il est possible de couper sa démonstration en morceaux en indiquant à l'environnement la valeur du 1\ier{} numéro de justification via la clé \verb+start+ : la valeur spéciale \verb+last+ indique de continuer la numérotation à la suite.

\begin{latexex}
\begin{demotab}[start = last]
    \demostep
        Axiome 3
            & $(A \wedge B) \implies C$
    \demostep
        m.p. sur \explref*{4} 
              et \explref*{5}
            & $C$
\end{demotab}
\end{latexex}


% ---------------------- %


\newparaexample{Référencer une indication}

L'argument optionnel de \macro{demostep} permet de définir un label qui ensuite facilitera le référencement d'une justification de façon pérenne via la macro non étoilée \macro{explref}.

\begin{latexex}
\begin{demotab}
    \demostep[demo-my-hyp]
        Hypothèse & $A$     
    \demostep[demo-axiom-1]
        Axiome 1  & $A \implies B$
    \demostep
        m.p. sur \explref{demo-my-hyp}
              et \explref{demo-axiom-1}
                  & $B$
\end{demotab}
\end{latexex}


\begin{remark}
    Prendre bien garde au fait que ce mécanisme utilise les macros \macro{label} et \macro{ref} de \LaTeX.
    On travaille donc avec des références globalement au document compilé.
\end{remark}


% ---------------------- %


\newparaexample{Indiquer ce que l'on cherche à faire}

Les clés optionnelles \verb+hyps+ pour plusieurs hypothèses, \verb+hyp+ pour une seule hypothèse et \verb+ccl+ pour la conclusion permettent d'expliquer ce que l'on démontre et sous quel contexte.

\begin{latexex}
\begin{demotab}[hyp = $A$, ccl = $B$]
    \demostep
        Hypothèse & $A$     
    \demostep
        Axiome 1  & $A \implies B$
    \demostep
        m.p. sur \explref*{1} 
              et \explref*{2}
                  & $B$
\end{demotab}
\end{latexex}


\begin{remark}
    Aucune des clés \verb+hyps+, \verb+hyp+ et \verb+ccl+ n'est obligatoire.
    Par contre il n'est pas possible d'utiliser à la fois les clés \verb+hyps+ et \verb+hyp+.
\end{remark}


% ---------------------- %


\newparaexample{Un tableau incomplet à remplir}

On peut proposer un tableau incomplet comme suit en utilisant obligatoirement \verb#\demostep{}# pour une étape non référencée dont la 1\iere{} cellule est vide
\footnote{
	Ceci est dû au mécanisme permetant de taper au choix \texttt{\textbackslash{}demostep[une-ref]} ou \texttt{\textbackslash{}demostep}. 
}.

\begin{latexex}
\begin{demotab}
    \demostep
        Hypothèse &     
    \demostep{} % --> Ne pas oublier !
                  & $A \implies B$
    \demostep{} % --> Ne pas oublier !
                  &
\end{demotab}
\end{latexex}


\begin{remark}
   Dans la section \ref{tnslog-tab-hard-proof-for-youngs} est présentée la macro \macro{explnothing} pour montrer qu'il n'y a rien à indiquer dans une cellule (son usage semble peu pertinent ici mais on ne sait jamais).
\end{remark}



% ---------------------- %


\newparaexample{Gérer les hauteurs des lignes}

Si l'exemple précédent est donné comme exercice demandant de remplir directement sur un énoncé les cases vides, il faut pouvoir modifier les hauteurs des lignes ponctuellement. L'exemple suivant montre comment faire ceci via la macro \macro{demoxspace} où le \prefix{x} est pour \prefix{extra}.
Cette macro a un argument optionnel permettant d'augmenter l'espacement vertical.

\begin{latexex}
A remplir...

\begin{demotab}
    \demostep
        Hypothèse &
    \demoxspace   
    \demostep{} % --> Ne pas oublier !
                  & $A \implies B$
    \demoxspace[1cm]
    \demostep{} % --> Ne pas oublier !
                  &
    \demoxspace[1cm]
\end{demotab}

Est-ce rempli ?
\end{latexex}


% ---------------------- %


\newparaexample{Gérer la largeur de la colonne centrale}

On constate que dans l'exemple précédent nous avons aussi besoin d'augmenter la largeur de la colonne centrale
\footnote{
   Il ne semble pas pertinent de proposer la possibilité de modifier les largeurs des colonnes non centrales.
}
pour laisser de la place à une rédaction manuscrite. Ceci se fait via l'option \verb#cwidth# de l'environnement. Voici une version améliorée du tableau précédent.

\begin{latexex-flat}
A remplir...

\begin{demotab}[cwidth = 7cm]
    \demostep
        Hypothèse &
    \demoxspace   
    \demostep{} % --> Ne pas oublier !
                  & $A \implies B$
    \demoxspace[1cm]
    \demostep{} % --> Ne pas oublier !
                  &
    \demoxspace[1cm]
\end{demotab}

Est-ce rempli ?
\end{latexex-flat}


% ---------------------- %


\subsection{Un tableau sur plusieurs pages}

Un tableau devant utiliser plusieurs pages sera scindé comme dans l'image page \pageref{tnslog:demotab-splitted} sans perte d'information
\footnote{
	Tout le travail est fait par l'environnement \env{longtable} du package éponyme.
}.

\begin{figure}[hbt!]\label{tnslog:demotab-splitted}
	\centering
	\frame{\includegraphics[scale = .5]{images/demotab-univ-broken[fr].png}}
\end{figure}

\end{document}
