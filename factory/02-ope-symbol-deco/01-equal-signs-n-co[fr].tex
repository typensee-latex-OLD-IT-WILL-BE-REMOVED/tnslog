\documentclass[12pt,a4paper]{article}

\makeatletter
	\usepackage[utf8]{inputenc}
\usepackage[T1]{fontenc}
\usepackage{ucs}

\usepackage[french]{babel,varioref}

\usepackage[top=2cm, bottom=2cm, left=1.5cm, right=1.5cm]{geometry}
\usepackage{enumitem}

\usepackage{pgffor}

\usepackage{multicol}

\usepackage{makecell}

\usepackage{color}
\usepackage{hyperref}
\hypersetup{
    colorlinks,
    citecolor=black,
    filecolor=black,
    linkcolor=black,
    urlcolor=black
}

\usepackage{amsthm}

\usepackage{tcolorbox}
\tcbuselibrary{listingsutf8}

\usepackage{ifplatform}

\usepackage{ifthen}

\usepackage{macroenvsign}


% Sections numbering

%\renewcommand\thechapter{\Alph{chapter}.}
\renewcommand\thesection{\Roman{section}.}
\renewcommand\thesubsection{\arabic{subsection}.}
\renewcommand\thesubsubsection{\roman{subsubsection}.}



% MISC

\newtcblisting{latexex}{%
	sharp corners,%
	left=1mm, right=1mm,%
	bottom=1mm, top=1mm,%
	colupper=red!75!blue,%
	listing side text
}

\newtcbinputlisting{\inputlatexex}[2][]{%
	listing file={#2},%
	sharp corners,%
	left=1mm, right=1mm,%
	bottom=1mm, top=1mm,%
	colupper=red!75!blue,%
	listing side text
}


\newtcblisting{latexex-flat}{%
	sharp corners,%
	left=1mm, right=1mm,%
	bottom=1mm, top=1mm,%
	colupper=red!75!blue,%
}

\newtcbinputlisting{\inputlatexexflat}[2][]{%
	listing file={#2},%
	sharp corners,%
	left=1mm, right=1mm,%
	bottom=1mm, top=1mm,%
	colupper=red!75!blue,%
}


\newtcblisting{latexex-alone}{%
	sharp corners,%
	left=1mm, right=1mm,%
	bottom=1mm, top=1mm,%
	colupper=red!75!blue,%
	listing only
}

\newtcbinputlisting{\inputlatexexalone}[2][]{%
	listing file={#2},%
	sharp corners,%
	left=1mm, right=1mm,%
	bottom=1mm, top=1mm,%
	colupper=red!75!blue,%
	listing only
}


\newcommand\inputlatexexcodeafter[1]{%
	\begin{center}
		\input{#1}
	\end{center}

	\vspace{-.5em}
	
	Le rendu précédent a été obtenu via le code suivant.
	
	\inputlatexexalone{#1}
}


\newcommand\inputlatexexcodebefore[1]{%
	\inputlatexexalone{#1}
	\vspace{-.75em}
	\begin{center}
		\textit{\footnotesize Rendu du code précédent}
		
		\medskip
		
		\input{#1}
	\end{center}
}


\newcommand\env[1]{\texttt{#1}}
\newcommand\macro[1]{\env{\textbackslash{}#1}}



\setlength{\parindent}{0cm}
\setlist{noitemsep}

\theoremstyle{definition}
\newtheorem*{remark}{Remarque}

\usepackage[raggedright]{titlesec}

\titleformat{\paragraph}[hang]{\normalfont\normalsize\bfseries}{\theparagraph}{1em}{}
\titlespacing*{\paragraph}{0pt}{3.25ex plus 1ex minus .2ex}{0.5em}


\newcommand\separation{
	\medskip
	\hfill\rule{0.5\textwidth}{0.75pt}\hfill
	\medskip
}


\newcommand\extraspace{
	\vspace{0.25em}
}


\newcommand\whyprefix[2]{%
	\textbf{\prefix{#1}}-#2%
}

\newcommand\mwhyprefix[2]{%
	\texttt{#1 = #1-#2}%
}

\newcommand\prefix[1]{%
	\texttt{#1}%
}


\newcommand\inenglish{\@ifstar{\@inenglish@star}{\@inenglish@no@star}}

\newcommand\@inenglish@star[1]{%
	\emph{\og #1 \fg}%
}

\newcommand\@inenglish@no@star[1]{%
	\@inenglish@star{#1} en anglais%
}


\newcommand\ascii{\texttt{ASCII}}


% Example
\newcounter{paraexample}[subsubsection]

\newcommand\@newexample@abstract[2]{%
	\paragraph{%
		#1%
		\if\relax\detokenize{#2}\relax\else {} -- #2\fi%
	}%
}



\newcommand\newparaexample{\@ifstar{\@newparaexample@star}{\@newparaexample@no@star}}

\newcommand\@newparaexample@no@star[1]{%
	\refstepcounter{paraexample}%
	\@newexample@abstract{Exemple \theparaexample}{#1}%
}

\newcommand\@newparaexample@star[1]{%
	\@newexample@abstract{Exemple}{#1}%
}


% Change log
\newcommand\topic{\@ifstar{\@topic@star}{\@topic@no@star}}

\newcommand\@topic@no@star[1]{%
	\textbf{\textsc{#1}.}%
}

\newcommand\@topic@star[1]{%
	\textbf{\textsc{#1} :}%
}



	\usepackage{01-special-ope}
\makeatother


\begin{document}

\section{Personnaliser les opérateurs de comparaison}

\subsection{Décorer avec du texte}

D'un point de vue pédagogique il peut être intéressant de disposer de différentes façons d'écrire une égalité, une non égalité ou une inégalité en lui ajoutant un texte descriptif.
Bien entendu on tord les règles de typographie avec ce type de pratique mais c'est pour le bien de la communauté éducative. Pour cela on utilisera l'une des macros suivantes ou leur version négative obtenue en préfixant le nom d'un \verb#n#.

\begin{enumerate}
	\item \macro{eq} donne $\eq$ 
	      tandis que
	      \macro{neq} donne $\neq$.

	\item \macro{less} et \macro{gtr} donnent $\less$ et $\gtr$
	      tandis que\macro{nless} et \macro{ngtr} donnent $\nless$ et $\ngtr$ .

	\item \macro{leq} et \macro{geq} donnent $\leq$ et $\geq$
	      tandis que\macro{nleq} et \macro{ngeq} donnent $\nleq$ et $\ngeq$ .
\end{enumerate}


\medskip


Voici un exemple où l'utilisation d'arguments optionnels permet de décorer des opérateurs.

\begin{latexex}
   $f(x) \eq[def]   x^3 + 1$
si $x    \leq[cond] 2$

    $f(x) \neq[cons] x^3 + 1$
car $x    \nleq[hyp] 2$
\end{latexex}

\begin{remark}
	La mise en forme du texte est faite par la macro \macro{txtdecoope} qui utilise \macro{coldecoope} pour la coloration \emph{(il est donc facile de changer la couleur du texte mais aussi la mise en forme du texte si besoin)}.
\end{remark}


% ---------------------- %


\subsection{Quelques nouvelles écritures symboliques}

Certaines décorations fournissent une écriture symbolique obtenue via la version étoilée de la macro d'un opérateur.
Si une telle écriture n'existe pas, le package utilisera silencieusement la version non étoilée.
Pour le moment, seule la macro \macro{equ} et sa version négative proposent ceci. Voici ce que cela donne.

\begin{latexex}
$a \eq*[def] 1$  ou  $a \neq*[def] 1$

$b \eq*[id]  2$  ou  $b \neq*[id]  2$
\end{latexex}


% ---------------------- %


\section{Fiches techniques}

\IDmacro[o]{eq }{1}

\IDmacro[o]{neq}{1}

\IDoption{} un texte de décoration.


\separation




\IDmacro[o]{eq* }{1}

\IDmacro[o]{neq*}{1}

\IDoption{} un texte de décoration ou une version symbolique si l'on utilise l'un des textes suivants.
%\begin{multicols}{5}
    \begin{enumerate}
    	\item \verb#def#
    
    	\item \verb#id#
    \end{enumerate}
%\end{multicols}


\separation


\begin{multicols}{4}
	\foreach \k in {less, leq, gtr, geq}{
	
		\IDmacro[o]{\k{}  }{1}
		
		\IDmacro[o]{\k{}* }{1}
		
		\smallskip
		
		\IDmacro[o]{n\k }{1}
		
		\IDmacro[o]{n\k*}{1}
	}
\end{multicols}

\vspace{-.75em}

\IDoption{} un texte de décoration.

\end{document}
