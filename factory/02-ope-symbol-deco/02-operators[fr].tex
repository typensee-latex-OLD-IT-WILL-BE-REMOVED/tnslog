\documentclass[12pt,a4paper]{article}

\makeatletter
    \usepackage[utf8]{inputenc}
\usepackage[T1]{fontenc}
\usepackage{ucs}

\usepackage[french]{babel,varioref}

\usepackage[top=2cm, bottom=2cm, left=1.5cm, right=1.5cm]{geometry}
\usepackage{enumitem}

\usepackage{pgffor}

\usepackage{multicol}

\usepackage{makecell}

\usepackage{color}
\usepackage{hyperref}
\hypersetup{
    colorlinks,
    citecolor=black,
    filecolor=black,
    linkcolor=black,
    urlcolor=black
}

\usepackage{amsthm}

\usepackage{tcolorbox}
\tcbuselibrary{listingsutf8}

\usepackage{ifplatform}

\usepackage{ifthen}

\usepackage{macroenvsign}


% Sections numbering

%\renewcommand\thechapter{\Alph{chapter}.}
\renewcommand\thesection{\Roman{section}.}
\renewcommand\thesubsection{\arabic{subsection}.}
\renewcommand\thesubsubsection{\roman{subsubsection}.}



% MISC

\newtcblisting{latexex}{%
	sharp corners,%
	left=1mm, right=1mm,%
	bottom=1mm, top=1mm,%
	colupper=red!75!blue,%
	listing side text
}

\newtcbinputlisting{\inputlatexex}[2][]{%
	listing file={#2},%
	sharp corners,%
	left=1mm, right=1mm,%
	bottom=1mm, top=1mm,%
	colupper=red!75!blue,%
	listing side text
}


\newtcblisting{latexex-flat}{%
	sharp corners,%
	left=1mm, right=1mm,%
	bottom=1mm, top=1mm,%
	colupper=red!75!blue,%
}

\newtcbinputlisting{\inputlatexexflat}[2][]{%
	listing file={#2},%
	sharp corners,%
	left=1mm, right=1mm,%
	bottom=1mm, top=1mm,%
	colupper=red!75!blue,%
}


\newtcblisting{latexex-alone}{%
	sharp corners,%
	left=1mm, right=1mm,%
	bottom=1mm, top=1mm,%
	colupper=red!75!blue,%
	listing only
}

\newtcbinputlisting{\inputlatexexalone}[2][]{%
	listing file={#2},%
	sharp corners,%
	left=1mm, right=1mm,%
	bottom=1mm, top=1mm,%
	colupper=red!75!blue,%
	listing only
}


\newcommand\inputlatexexcodeafter[1]{%
	\begin{center}
		\input{#1}
	\end{center}

	\vspace{-.5em}
	
	Le rendu précédent a été obtenu via le code suivant.
	
	\inputlatexexalone{#1}
}


\newcommand\inputlatexexcodebefore[1]{%
	\inputlatexexalone{#1}
	\vspace{-.75em}
	\begin{center}
		\textit{\footnotesize Rendu du code précédent}
		
		\medskip
		
		\input{#1}
	\end{center}
}


\newcommand\env[1]{\texttt{#1}}
\newcommand\macro[1]{\env{\textbackslash{}#1}}



\setlength{\parindent}{0cm}
\setlist{noitemsep}

\theoremstyle{definition}
\newtheorem*{remark}{Remarque}

\usepackage[raggedright]{titlesec}

\titleformat{\paragraph}[hang]{\normalfont\normalsize\bfseries}{\theparagraph}{1em}{}
\titlespacing*{\paragraph}{0pt}{3.25ex plus 1ex minus .2ex}{0.5em}


\newcommand\separation{
	\medskip
	\hfill\rule{0.5\textwidth}{0.75pt}\hfill
	\medskip
}


\newcommand\extraspace{
	\vspace{0.25em}
}


\newcommand\whyprefix[2]{%
	\textbf{\prefix{#1}}-#2%
}

\newcommand\mwhyprefix[2]{%
	\texttt{#1 = #1-#2}%
}

\newcommand\prefix[1]{%
	\texttt{#1}%
}


\newcommand\inenglish{\@ifstar{\@inenglish@star}{\@inenglish@no@star}}

\newcommand\@inenglish@star[1]{%
	\emph{\og #1 \fg}%
}

\newcommand\@inenglish@no@star[1]{%
	\@inenglish@star{#1} en anglais%
}


\newcommand\ascii{\texttt{ASCII}}


% Example
\newcounter{paraexample}[subsubsection]

\newcommand\@newexample@abstract[2]{%
	\paragraph{%
		#1%
		\if\relax\detokenize{#2}\relax\else {} -- #2\fi%
	}%
}



\newcommand\newparaexample{\@ifstar{\@newparaexample@star}{\@newparaexample@no@star}}

\newcommand\@newparaexample@no@star[1]{%
	\refstepcounter{paraexample}%
	\@newexample@abstract{Exemple \theparaexample}{#1}%
}

\newcommand\@newparaexample@star[1]{%
	\@newexample@abstract{Exemple}{#1}%
}


% Change log
\newcommand\topic{\@ifstar{\@topic@star}{\@topic@no@star}}

\newcommand\@topic@no@star[1]{%
	\textbf{\textsc{#1}.}%
}

\newcommand\@topic@star[1]{%
	\textbf{\textsc{#1} :}%
}


	\usepackage{01-equal-signs-n-co}

    \usepackage{02-operators}
\makeatother


\begin{document}

\section{Équivalences et implications}

\subsection{Des symboles supplémentaires}

\newparaexample{Implication réciproque}

En plus des opérateurs \macro{iff} et \macro{implies} proposés par \LaTeX{}, il a été ajouté l'opérateur \macro{liesimp}, où l'on a inversé les groupes syllabiques de \macro{implies}, un opérateur pour pour obtenir $\liesimp$
\footnote{
	Penser aussi aux preuves d'équivalence par double implication.
},
ainsi que des versions négatives. Voici un exemple d'utilisation.

\begin{latexex}
$(A \implies B)
 \iff (B \liesimp A)$

$(A \implies B)
 \niff (A \nimplies B)$
\end{latexex}


% ---------------------- %


\newparaexample{Des opérateurs décorés}

Tout comme pour les comparaisons, il existe des versions décorées de type test, hypothèse, condition \dots{} 
Elles sont toutes présentes dans l'exemple suivant.

\begin{latexex}
$A \iffappli B \niffchoice C$

$A \impliescond B \nimpliescons C$

$A \liesimphyp B \nliesimptest C$
\end{latexex}


\subsection{Une table récapitulative}

La table \ref{table:decorations-operators} \vpageref{table:decorations-operators} montre toutes les associations autorisées entre opérateurs logiques et décorations.


% ---------------------- %


\subsection{Fiches techniques}

\paragraph{Opérateurs décorés -- Pour la logique}

% == Decorated versions - START == %

\foreach \k in {iff, iffappli, iffchoice, iffcond, iffcons, iffhyp, ifftest}{
	\IDmacro*{\k}{0}

}
    
\separation

\foreach \k in {niff, niffappli, niffchoice, niffcond, niffcons, niffhyp, nifftest}{
	\IDmacro*{\k}{0}

}
    
\separation

\foreach \k in {implies, impliesappli, implieschoice, impliescond, impliescons, implieshyp, impliestest}{
	\IDmacro*{\k}{0}

}
    
\separation

\foreach \k in {nimplies, nimpliesappli, nimplieschoice, nimpliescond, nimpliescons, nimplieshyp, nimpliestest}{
	\IDmacro*{\k}{0}

}
    
\separation

\foreach \k in {liesimp, liesimpappli, liesimpchoice, liesimpcond, liesimpcons, liesimphyp, liesimptest}{
	\IDmacro*{\k}{0}

}
    
\separation

\foreach \k in {nliesimp, nliesimpappli, nliesimpchoice, nliesimpcond, nliesimpcons, nliesimphyp, nliesimptest}{
	\IDmacro*{\k}{0}

}
    
% == Decorated versions - END == %


% ---------------------- %


\subsection{Équivalences et implications verticales}

\paragraph{À quoi cela sert-il ?}

Les sections \ref{explain-proof-for-youngs} et \ref{explain-proof} présentent deux environnements pour détailler les étapes d'un raisonnement.
Avec ces outils il devient utile d'avoir des versions verticales non décorées des symboles d'équivalence et d'implication. Voici comment les obtenir \emph{(tous les cas possibles ont été indiqués)}.
Bien entendu le préfixe \prefix{v} est pour \whyprefix{v}{ertical}.

\begin{latexex}
\begin{tabular}{cccccc}
    $A$          & $B$
  & $C$          & $D$
  & $E$          & $F$
  \\
    $\viff$      & $\vimplies$   
  & $\vliesimp$  & $\nviff$
  & $\nvimplies$ & $\nvliesimp$
  \\
    $A$          & $B$
  & $C$          & $D$
  & $E$          & $F$
\end{tabular}
\end{latexex}


% ---------------------- %


\subsection{Fiches techniques}

\paragraph{Opérateurs de logique \og verticaux \fg}

% == Vertical versions - START == %

\foreach \k in {viff, vimplies, vliesimp}{

	\IDmacro*{\k}{0}

	\IDmacro*{n\k}{0}

    \extraspace
}

% == Vertical versions - END == %

\end{document}
