\documentclass[12pt,a4paper]{article}

\makeatletter
    \usepackage[utf8]{inputenc}
\usepackage[T1]{fontenc}
\usepackage{ucs}

\usepackage[french]{babel,varioref}

\usepackage[top=2cm, bottom=2cm, left=1.5cm, right=1.5cm]{geometry}
\usepackage{enumitem}

\usepackage{pgffor}

\usepackage{multicol}

\usepackage{makecell}

\usepackage{color}
\usepackage{hyperref}
\hypersetup{
    colorlinks,
    citecolor=black,
    filecolor=black,
    linkcolor=black,
    urlcolor=black
}

\usepackage{amsthm}

\usepackage{tcolorbox}
\tcbuselibrary{listingsutf8}

\usepackage{ifplatform}

\usepackage{ifthen}

\usepackage{macroenvsign}


% Sections numbering

%\renewcommand\thechapter{\Alph{chapter}.}
\renewcommand\thesection{\Roman{section}.}
\renewcommand\thesubsection{\arabic{subsection}.}
\renewcommand\thesubsubsection{\roman{subsubsection}.}



% MISC

\newtcblisting{latexex}{%
	sharp corners,%
	left=1mm, right=1mm,%
	bottom=1mm, top=1mm,%
	colupper=red!75!blue,%
	listing side text
}

\newtcbinputlisting{\inputlatexex}[2][]{%
	listing file={#2},%
	sharp corners,%
	left=1mm, right=1mm,%
	bottom=1mm, top=1mm,%
	colupper=red!75!blue,%
	listing side text
}


\newtcblisting{latexex-flat}{%
	sharp corners,%
	left=1mm, right=1mm,%
	bottom=1mm, top=1mm,%
	colupper=red!75!blue,%
}

\newtcbinputlisting{\inputlatexexflat}[2][]{%
	listing file={#2},%
	sharp corners,%
	left=1mm, right=1mm,%
	bottom=1mm, top=1mm,%
	colupper=red!75!blue,%
}


\newtcblisting{latexex-alone}{%
	sharp corners,%
	left=1mm, right=1mm,%
	bottom=1mm, top=1mm,%
	colupper=red!75!blue,%
	listing only
}

\newtcbinputlisting{\inputlatexexalone}[2][]{%
	listing file={#2},%
	sharp corners,%
	left=1mm, right=1mm,%
	bottom=1mm, top=1mm,%
	colupper=red!75!blue,%
	listing only
}


\newcommand\inputlatexexcodeafter[1]{%
	\begin{center}
		\input{#1}
	\end{center}

	\vspace{-.5em}
	
	Le rendu précédent a été obtenu via le code suivant.
	
	\inputlatexexalone{#1}
}


\newcommand\inputlatexexcodebefore[1]{%
	\inputlatexexalone{#1}
	\vspace{-.75em}
	\begin{center}
		\textit{\footnotesize Rendu du code précédent}
		
		\medskip
		
		\input{#1}
	\end{center}
}


\newcommand\env[1]{\texttt{#1}}
\newcommand\macro[1]{\env{\textbackslash{}#1}}



\setlength{\parindent}{0cm}
\setlist{noitemsep}

\theoremstyle{definition}
\newtheorem*{remark}{Remarque}

\usepackage[raggedright]{titlesec}

\titleformat{\paragraph}[hang]{\normalfont\normalsize\bfseries}{\theparagraph}{1em}{}
\titlespacing*{\paragraph}{0pt}{3.25ex plus 1ex minus .2ex}{0.5em}


\newcommand\separation{
	\medskip
	\hfill\rule{0.5\textwidth}{0.75pt}\hfill
	\medskip
}


\newcommand\extraspace{
	\vspace{0.25em}
}


\newcommand\whyprefix[2]{%
	\textbf{\prefix{#1}}-#2%
}

\newcommand\mwhyprefix[2]{%
	\texttt{#1 = #1-#2}%
}

\newcommand\prefix[1]{%
	\texttt{#1}%
}


\newcommand\inenglish{\@ifstar{\@inenglish@star}{\@inenglish@no@star}}

\newcommand\@inenglish@star[1]{%
	\emph{\og #1 \fg}%
}

\newcommand\@inenglish@no@star[1]{%
	\@inenglish@star{#1} en anglais%
}


\newcommand\ascii{\texttt{ASCII}}


% Example
\newcounter{paraexample}[subsubsection]

\newcommand\@newexample@abstract[2]{%
	\paragraph{%
		#1%
		\if\relax\detokenize{#2}\relax\else {} -- #2\fi%
	}%
}



\newcommand\newparaexample{\@ifstar{\@newparaexample@star}{\@newparaexample@no@star}}

\newcommand\@newparaexample@no@star[1]{%
	\refstepcounter{paraexample}%
	\@newexample@abstract{Exemple \theparaexample}{#1}%
}

\newcommand\@newparaexample@star[1]{%
	\@newexample@abstract{Exemple}{#1}%
}


% Change log
\newcommand\topic{\@ifstar{\@topic@star}{\@topic@no@star}}

\newcommand\@topic@no@star[1]{%
	\textbf{\textsc{#1}.}%
}

\newcommand\@topic@star[1]{%
	\textbf{\textsc{#1} :}%
}



    \usepackage{01-calc-explained}
\makeatother



\begin{document}

%\section{Détailler un raisonnement simple}

\subsection{De courts commentaires}

\newparaexample{Sans alignement}

Il est possible d'ajouter de petits commentaires via \macro{comthis} où \prefix{comthis} est pour \whyprefix{com}{ment} \prefix{this} soit \inenglish{commenter ceci}.
 
\begin{latexex}
\begin{explain}
    (a + b)^2
        \comthis{Forme facto.}
        \explnext*{Id.Rq. -- Dév.}%
                  {Id.Rq. -- Facto.}
    a^2 + 2 a b + b^2
        \comthis{Forme dév.}
\end{explain}
\end{latexex}


\begin{remark}
	La mise en forme du texte des commentaires est fait via la macro personnalisable \macro{explcom}.
	Quant à l'espacement ajouté entre le texte et son commentaire il est défini par la macro \macro{expltxtspacein} qui est égale à \verb+2em+ par défaut.
\end{remark}


% ---------------------- %


\newparaexample{Tout aligner}

Il peut être utile d'aligner tous les commentaires. Ceci s'obtient via l'option \verb+com = al+ où \prefix{al} est pour \whyprefix{al}{igné} \emph{(par défaut \prefix{com = nal} avec le préfixe \prefix{n} pour \whyprefix{n}{on})}.

\begin{latexex}
\begin{explain}[com = al]
    (a + b)^2
        \comthis{Forme facto.}
        \explnext*{Id.Rq - Dév.}%
                  {Id.Rq - Facto.}
    a^2 + 2 a b + b^2
        \comthis{Forme dév.}
\end{explain}
\end{latexex}


% ---------------------- %


\newparaexample{Le meilleur des deux mondes}

Dans d'autres situations, utilisez les deux types d'alignement peut faire sens. Ceci s'obtient via l'option \verb+com = al+ et l'emploi de la macro étoilée \macro{comthis*} à chaque fois que l'on souhaite "coller" un commentaire le plus à gauche possible.

\begin{latexex}
\begin{explain}[com = al]
    (a + b) (a + b)
        \comthis{Forme facto.}
        \explnext{Via $x^2 = x \cdot x$.}
    (a + b)^2
        \comthis*{Au passage...}
        \explnext*{Id.Rq - Dév.}%
                  {Id.Rq - Facto.}
    a^2 + 2 a b + b^2
        \comthis{Forme dév.}
\end{explain}
\end{latexex}


\begin{remark}
	Si l'alignement n'est pas activé, les macros \macro{comthis*} et \macro{comthis} auront toutes les deux le même effet.
\end{remark}


% ---------------------- %


\newparaexample{Ceci marche aussi avec le style \og fléché \fg}

Voici ce que donne le mode mixte lorsque des flèches sont utilisées pour les explications. Il semble moins pertinent ici de mixer les modes \og alignement \fg{} et \og non alignement \fg{} mais chacun pris séparément peut avoir son utilité.

\begin{latexex-flat}
\begin{explain}[style = ar, com = al]
    (a + b) (a + b)
        \comthis{Forme facto.}
        \explnext{Via $x^2 = x \cdot x$.}
    (a + b)^2
        \comthis*{Au passage...}
        \explnext*{Id.Rq - Dév.}%
                  {Id.Rq - Facto.}
    a^2 + 2 a b + b^2
        \comthis{Forme dév.}
\end{explain}
\end{latexex-flat}


\begin{remark}
	Bien entendu il est impossible de commenter le tout début en mode fléché court.
\end{remark}


\end{document}