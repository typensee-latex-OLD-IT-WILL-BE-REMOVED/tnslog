\documentclass[12pt,a4paper]{article}

\makeatletter
    \usepackage[utf8]{inputenc}
\usepackage[T1]{fontenc}
\usepackage{ucs}

\usepackage[french]{babel,varioref}

\usepackage[top=2cm, bottom=2cm, left=1.5cm, right=1.5cm]{geometry}
\usepackage{enumitem}

\usepackage{pgffor}

\usepackage{multicol}

\usepackage{makecell}

\usepackage{color}
\usepackage{hyperref}
\hypersetup{
    colorlinks,
    citecolor=black,
    filecolor=black,
    linkcolor=black,
    urlcolor=black
}

\usepackage{amsthm}

\usepackage{tcolorbox}
\tcbuselibrary{listingsutf8}

\usepackage{ifplatform}

\usepackage{ifthen}

\usepackage{macroenvsign}


% Sections numbering

%\renewcommand\thechapter{\Alph{chapter}.}
\renewcommand\thesection{\Roman{section}.}
\renewcommand\thesubsection{\arabic{subsection}.}
\renewcommand\thesubsubsection{\roman{subsubsection}.}



% MISC

\newtcblisting{latexex}{%
	sharp corners,%
	left=1mm, right=1mm,%
	bottom=1mm, top=1mm,%
	colupper=red!75!blue,%
	listing side text
}

\newtcbinputlisting{\inputlatexex}[2][]{%
	listing file={#2},%
	sharp corners,%
	left=1mm, right=1mm,%
	bottom=1mm, top=1mm,%
	colupper=red!75!blue,%
	listing side text
}


\newtcblisting{latexex-flat}{%
	sharp corners,%
	left=1mm, right=1mm,%
	bottom=1mm, top=1mm,%
	colupper=red!75!blue,%
}

\newtcbinputlisting{\inputlatexexflat}[2][]{%
	listing file={#2},%
	sharp corners,%
	left=1mm, right=1mm,%
	bottom=1mm, top=1mm,%
	colupper=red!75!blue,%
}


\newtcblisting{latexex-alone}{%
	sharp corners,%
	left=1mm, right=1mm,%
	bottom=1mm, top=1mm,%
	colupper=red!75!blue,%
	listing only
}

\newtcbinputlisting{\inputlatexexalone}[2][]{%
	listing file={#2},%
	sharp corners,%
	left=1mm, right=1mm,%
	bottom=1mm, top=1mm,%
	colupper=red!75!blue,%
	listing only
}


\newcommand\inputlatexexcodeafter[1]{%
	\begin{center}
		\input{#1}
	\end{center}

	\vspace{-.5em}
	
	Le rendu précédent a été obtenu via le code suivant.
	
	\inputlatexexalone{#1}
}


\newcommand\inputlatexexcodebefore[1]{%
	\inputlatexexalone{#1}
	\vspace{-.75em}
	\begin{center}
		\textit{\footnotesize Rendu du code précédent}
		
		\medskip
		
		\input{#1}
	\end{center}
}


\newcommand\env[1]{\texttt{#1}}
\newcommand\macro[1]{\env{\textbackslash{}#1}}



\setlength{\parindent}{0cm}
\setlist{noitemsep}

\theoremstyle{definition}
\newtheorem*{remark}{Remarque}

\usepackage[raggedright]{titlesec}

\titleformat{\paragraph}[hang]{\normalfont\normalsize\bfseries}{\theparagraph}{1em}{}
\titlespacing*{\paragraph}{0pt}{3.25ex plus 1ex minus .2ex}{0.5em}


\newcommand\separation{
	\medskip
	\hfill\rule{0.5\textwidth}{0.75pt}\hfill
	\medskip
}


\newcommand\extraspace{
	\vspace{0.25em}
}


\newcommand\whyprefix[2]{%
	\textbf{\prefix{#1}}-#2%
}

\newcommand\mwhyprefix[2]{%
	\texttt{#1 = #1-#2}%
}

\newcommand\prefix[1]{%
	\texttt{#1}%
}


\newcommand\inenglish{\@ifstar{\@inenglish@star}{\@inenglish@no@star}}

\newcommand\@inenglish@star[1]{%
	\emph{\og #1 \fg}%
}

\newcommand\@inenglish@no@star[1]{%
	\@inenglish@star{#1} en anglais%
}


\newcommand\ascii{\texttt{ASCII}}


% Example
\newcounter{paraexample}[subsubsection]

\newcommand\@newexample@abstract[2]{%
	\paragraph{%
		#1%
		\if\relax\detokenize{#2}\relax\else {} -- #2\fi%
	}%
}



\newcommand\newparaexample{\@ifstar{\@newparaexample@star}{\@newparaexample@no@star}}

\newcommand\@newparaexample@no@star[1]{%
	\refstepcounter{paraexample}%
	\@newexample@abstract{Exemple \theparaexample}{#1}%
}

\newcommand\@newparaexample@star[1]{%
	\@newexample@abstract{Exemple}{#1}%
}


% Change log
\newcommand\topic{\@ifstar{\@topic@star}{\@topic@no@star}}

\newcommand\@topic@no@star[1]{%
	\textbf{\textsc{#1}.}%
}

\newcommand\@topic@star[1]{%
	\textbf{\textsc{#1} :}%
}


    % == PACKAGES USED == %

\RequirePackage{mathtools}
\RequirePackage{centernot}

\RequirePackage{graphicx}

\RequirePackage{tnscom}


% == DEFINITIONS == %

% == Vertical versions == %

\newcommand\viff{\mathrel{\Updownarrow}}
\newcommand\vimplies{\mathrel{\Downarrow}}
\newcommand\vbecauseof{\mathrel{\Uparrow}}

\newcommand\nviff{\centernot\viff}
\newcommand\nvimplies{\centernot\vimplies}
\newcommand\nvbecauseof{\centernot\vbecauseof}


% == Decorations - START == %

% Source for the short sysmbols.
%    * https://tex.stackexchange.com/a/585267/6880

\newcommand\shorteq{\mathrel{\mathpalette\shorteq@{.55}}}
\newcommand{\shorteq@}[2]{%
  \resizebox{#2\width}{\height}{$\m@th#1=$}%
}

\newcommand\shortless{\mathrel{\mathpalette\shortless@{.55}}}
\newcommand{\shortless@}[2]{%
  \resizebox{#2\width}{\height}{$\m@th#1<$}%
}

\newcommand\shortgtr{\mathrel{\mathpalette\shortgtr@{.55}}}
\newcommand{\shortgtr@}[2]{%
  \resizebox{#2\width}{\height}{$\m@th#1>$}%
}

% Decorable operators.

\newcommand\coldecoope{blue}

\newcommand\txtdecoope[1]{%
	\text{\tiny\color{\coldecoope}#1}%
}

\newcommand\eq{\@ifstar{\tnslog@eq@star}{\tnslog@eq@no@star}}

\newcommand\tnslog@eq@no@star[1][]{%
    \if\relax\detokenize{#1}\relax%
    	=%
    \else%
        \tns@over@math@symbol{\txtdecoope{#1}}{=}%
    \fi%
}

\newcommand\tnslog@eq@star[1][]{%
    \if\relax\detokenize{#1}\relax%
    	=%
    \else%
        \IfEqCase{#1}{%
            {def}{\coloneqq}%
            {id}{\rightleftharpoons}%
        }[%
            \tns@over@math@symbol{\txtdecoope{#1}}{=}%
        ]%
    \fi%
}

\let\oldneq\neq
\renewcommand\neq{\@ifstar{\tnslog@neq@star}{\tnslog@neq@no@star}}

\newcommand\tnslog@neq@no@star[1][]{%
    \if\relax\detokenize{#1}\relax%
    	\oldneq%
    \else%
        \tns@over@math@symbol{\txtdecoope{#1}}{\oldneq}%
    \fi%
}

\newcommand\tnslog@neq@star[1][]{%
    \if\relax\detokenize{#1}\relax%
    	\oldneq%
    \else%
        \IfEqCase{#1}{%
            {def}{\centernot\coloneqq}%
            {id}{\centernot\rightleftharpoons}%
        }[%
            \tns@over@math@symbol{\txtdecoope{#1}}{\oldneq}%
        ]%
    \fi%
}

\newcommand\less{\@ifstar{\tnslog@less@star}{\tnslog@less@no@star}}

\newcommand\tnslog@less@no@star[1][]{%
    \if\relax\detokenize{#1}\relax%
    	<%
    \else%
        \tns@over@math@symbol{\txtdecoope{#1}}{<}%
    \fi%
}

\newcommand\tnslog@less@star[1][]{%
    \if\relax\detokenize{#1}\relax%
    	<%
    \else%
        \tns@over@math@symbol{\txtdecoope{#1}}{<}%
    \fi%
}

\let\oldnless\nless
\renewcommand\nless{\@ifstar{\tnslog@nless@star}{\tnslog@nless@no@star}}

\newcommand\tnslog@nless@no@star[1][]{%
    \if\relax\detokenize{#1}\relax%
    	\oldnless%
    \else%
        \tns@over@math@symbol{\txtdecoope{#1}}{\oldnless}%
    \fi%
}

\newcommand\tnslog@nless@star[1][]{%
    \if\relax\detokenize{#1}\relax%
    	\oldnless%
    \else%
        \tns@over@math@symbol{\txtdecoope{#1}}{\oldnless}%
    \fi%
}

\newcommand\gtr{\@ifstar{\tnslog@gtr@star}{\tnslog@gtr@no@star}}

\newcommand\tnslog@gtr@no@star[1][]{%
    \if\relax\detokenize{#1}\relax%
    	>%
    \else%
        \tns@over@math@symbol{\txtdecoope{#1}}{>}%
    \fi%
}

\newcommand\tnslog@gtr@star[1][]{%
    \if\relax\detokenize{#1}\relax%
    	>%
    \else%
        \tns@over@math@symbol{\txtdecoope{#1}}{>}%
    \fi%
}

\let\oldngtr\ngtr
\renewcommand\ngtr{\@ifstar{\tnslog@ngtr@star}{\tnslog@ngtr@no@star}}

\newcommand\tnslog@ngtr@no@star[1][]{%
    \if\relax\detokenize{#1}\relax%
    	\oldngtr%
    \else%
        \tns@over@math@symbol{\txtdecoope{#1}}{\oldngtr}%
    \fi%
}

\newcommand\tnslog@ngtr@star[1][]{%
    \if\relax\detokenize{#1}\relax%
    	\oldngtr%
    \else%
        \tns@over@math@symbol{\txtdecoope{#1}}{\oldngtr}%
    \fi%
}

\let\oldleq\leq
\renewcommand\leq{\@ifstar{\tnslog@leq@star}{\tnslog@leq@no@star}}

\newcommand\tnslog@leq@no@star[1][]{%
    \if\relax\detokenize{#1}\relax%
    	\oldleq%
    \else%
        \tns@over@math@symbol{\txtdecoope{#1}}{\oldleq}%
    \fi%
}

\newcommand\tnslog@leq@star[1][]{%
    \if\relax\detokenize{#1}\relax%
    	\oldleq%
    \else%
        \tns@over@math@symbol{\txtdecoope{#1}}{\oldleq}%
    \fi%
}

\let\oldnleq\nleq
\renewcommand\nleq{\@ifstar{\tnslog@nleq@star}{\tnslog@nleq@no@star}}

\newcommand\tnslog@nleq@no@star[1][]{%
    \if\relax\detokenize{#1}\relax%
    	\oldnleq%
    \else%
        \tns@over@math@symbol{\txtdecoope{#1}}{\oldnleq}%
    \fi%
}

\newcommand\tnslog@nleq@star[1][]{%
    \if\relax\detokenize{#1}\relax%
    	\oldnleq%
    \else%
        \tns@over@math@symbol{\txtdecoope{#1}}{\oldnleq}%
    \fi%
}

\let\oldgeq\geq
\renewcommand\geq{\@ifstar{\tnslog@geq@star}{\tnslog@geq@no@star}}

\newcommand\tnslog@geq@no@star[1][]{%
    \if\relax\detokenize{#1}\relax%
    	\oldgeq%
    \else%
        \tns@over@math@symbol{\txtdecoope{#1}}{\oldgeq}%
    \fi%
}

\newcommand\tnslog@geq@star[1][]{%
    \if\relax\detokenize{#1}\relax%
    	\oldgeq%
    \else%
        \tns@over@math@symbol{\txtdecoope{#1}}{\oldgeq}%
    \fi%
}

\let\oldngeq\ngeq
\renewcommand\ngeq{\@ifstar{\tnslog@ngeq@star}{\tnslog@ngeq@no@star}}

\newcommand\tnslog@ngeq@no@star[1][]{%
    \if\relax\detokenize{#1}\relax%
    	\oldngeq%
    \else%
        \tns@over@math@symbol{\txtdecoope{#1}}{\oldngeq}%
    \fi%
}

\newcommand\tnslog@ngeq@star[1][]{%
    \if\relax\detokenize{#1}\relax%
    	\oldngeq%
    \else%
        \tns@over@math@symbol{\txtdecoope{#1}}{\oldngeq}%
    \fi%
}

\let\oldiff\iff
\renewcommand\iff{\@ifstar{\tnslog@iff@star}{\tnslog@iff@no@star}}

\newcommand\tnslog@iff@no@star[1][]{%
    \if\relax\detokenize{#1}\relax%
    	\oldiff%
    \else%
        \tns@over@math@symbol{\txtdecoope{#1}}{\oldiff}%
    \fi%
}

\newcommand\tnslog@iff@star[1][]{%
    \if\relax\detokenize{#1}\relax%
    	\oldiff%
    \else%
        \tns@over@math@symbol{\txtdecoope{#1}}{\oldiff}%
    \fi%
}

\newcommand\niff{\@ifstar{\tnslog@niff@star}{\tnslog@niff@no@star}}

\newcommand\tnslog@niff@no@star[1][]{%
    \if\relax\detokenize{#1}\relax%
    	\centernot\iff%
    \else%
        \tns@over@math@symbol{\txtdecoope{#1}}{\centernot\iff}%
    \fi%
}

\newcommand\tnslog@niff@star[1][]{%
    \if\relax\detokenize{#1}\relax%
    	\centernot\iff%
    \else%
        \tns@over@math@symbol{\txtdecoope{#1}}{\centernot\iff}%
    \fi%
}

\let\oldimplies\implies
\renewcommand\implies{\@ifstar{\tnslog@implies@star}{\tnslog@implies@no@star}}

\newcommand\tnslog@implies@no@star[1][]{%
    \if\relax\detokenize{#1}\relax%
    	\oldimplies%
    \else%
        \tns@over@math@symbol{\txtdecoope{#1}}{\oldimplies}%
    \fi%
}

\newcommand\tnslog@implies@star[1][]{%
    \if\relax\detokenize{#1}\relax%
    	\oldimplies%
    \else%
        \tns@over@math@symbol{\txtdecoope{#1}}{\oldimplies}%
    \fi%
}

\newcommand\nimplies{\@ifstar{\tnslog@nimplies@star}{\tnslog@nimplies@no@star}}

\newcommand\tnslog@nimplies@no@star[1][]{%
    \if\relax\detokenize{#1}\relax%
    	\centernot\implies%
    \else%
        \tns@over@math@symbol{\txtdecoope{#1}}{\centernot\implies}%
    \fi%
}

\newcommand\tnslog@nimplies@star[1][]{%
    \if\relax\detokenize{#1}\relax%
    	\centernot\implies%
    \else%
        \tns@over@math@symbol{\txtdecoope{#1}}{\centernot\implies}%
    \fi%
}

\newcommand\becauseof{\@ifstar{\tnslog@becauseof@star}{\tnslog@becauseof@no@star}}

\newcommand\tnslog@becauseof@no@star[1][]{%
    \if\relax\detokenize{#1}\relax%
    	\mathrel{\Longleftarrow}%
    \else%
        \tns@over@math@symbol{\txtdecoope{#1}}{\mathrel{\Longleftarrow}}%
    \fi%
}

\newcommand\tnslog@becauseof@star[1][]{%
    \if\relax\detokenize{#1}\relax%
    	\mathrel{\Longleftarrow}%
    \else%
        \tns@over@math@symbol{\txtdecoope{#1}}{\mathrel{\Longleftarrow}}%
    \fi%
}

\newcommand\nbecauseof{\@ifstar{\tnslog@nbecauseof@star}{\tnslog@nbecauseof@no@star}}

\newcommand\tnslog@nbecauseof@no@star[1][]{%
    \if\relax\detokenize{#1}\relax%
    	\centernot\becauseof%
    \else%
        \tns@over@math@symbol{\txtdecoope{#1}}{\centernot\becauseof}%
    \fi%
}

\newcommand\tnslog@nbecauseof@star[1][]{%
    \if\relax\detokenize{#1}\relax%
    	\centernot\becauseof%
    \else%
        \tns@over@math@symbol{\txtdecoope{#1}}{\centernot\becauseof}%
    \fi%
}

% == Decorations - END == %
    
    \usepackage{01-stepcalc}
\makeatother



\begin{document}

\section{Détailler un raisonnement simple}

\subsection{Version pour le lycée et après} \label{tnslog-stepcalc-proof}

\newparaexample{Avec les réglages par défaut}

L'environnement \env{stepcalc} permet de détailler les étapes principales d'un calcul ou d'un raisonnement simple en s'appuyant sur la macro \macro{explnext} dont le nom vient de \emph{\og \whyprefix{expl}{ain} \prefix{next} step \fg} soit \inenglish{expliquer la prochaine étape}
\footnote{
    Cet environnement utilise aussi le package \texttt{witharrows} qui est très sympathique pour expliquer des étapes de calcul.
}.
On dispose aussi de \macro{explnext*} pour des explications descendantes et/ou montantes 
\footnote{
    Les explications données ne doivent pas être trop longues car ce serait contre-productif.
}.

\medskip


Ci-dessous se trouve un exemple, très farfelu vers la fin, où l'on utilise les réglages par défaut.
Notons au passage que ce type de présentation n'est sûrement pas bien adaptée à un jeune public pour lequel une 2\ieme{} façon de détailler des calculs et/ou un raisonnement simple est proposée plus bas dans la section \ref{tnslog-stepcalc-proof-for-youngs}.

\begin{latexex-flat}
\begin{stepcalc}
    (a + b)^2
		\explnext{On utilise $x^2 = x \cdot x$.}
    (a + b) (a + b)
        \explnext*{Double développement depuis la parenthèse gauche.}%
                  {Double factorisation pas facile.}
    a^2 + a b + b a + b^2
        \explnext*{}%
                  {Commutativité du produit.}
    a^2 + 2 a b + b^2
        \explnext*{Commutativité de l'addition.}%
                  {}
    a^2 + b^2 + 2 a b
\end{stepcalc}
\end{latexex-flat}


\begin{remark}
    Il faut savoir que la mise en forme est celle d'une formule ce qui peut rendre service comme dans l'exemple suivant.

\begin{latexex}
Un calcul avec un placement pouvant être 
utile :
\begin{stepcalc}
    (a + b)^2
        \explnext{Identité remarquable.}
    a^2 + b^2 + 2 a b
\end{stepcalc}
\end{latexex}

Avec un retour à la ligne, il faudra donc si besoin gérer l'espacement vertical.

\begin{latexex}
Mon calcul pas trop proche.

\medskip
\begin{stepcalc}
    (a + b)^2
        \explnext{Identité remarquable.}
    a^2 + b^2 + 2 a b
\end{stepcalc}
\end{latexex}
\end{remark}


\begin{remark}
    Voici des petites choses à connaître sur les macros \macro{explnext} et \macro{explnext*}.
    \begin{enumerate}
        \item \macro{expltxt} est utilisée par \macro{explnext} pour mettre en forme le texte d'explication.

        \item \macro{expltxtup} et \macro{expltxtdown} sont utilisées par \macro{explnext*} décorer les textes d'explication juste avant leur mise en forme finale via \macro{expltxtupdown}.

        \item \macro{explnext} et \macro{explnext*} utilisent la macro constante \macro{expltxtspacein} pour l'espacement entre le symbole et la courte explication. Par défaut, cette macro vaut \verb+2em+.
    \end{enumerate}
 
\end{remark}


% ---------------------- %


\newparaexample{Utiliser un autre symbole globalement}

L'environnement \env{stepcalc} possède plusieurs options dont l'une est \verb+ope+ qui vaut \verb+{=}+ par défaut. Ceci permet de faire ce qui suit sans effort.

\begin{latexex}
\begin{stepcalc}[ope = \viff]
    x^2 + 10 x + 25 = 0
        \explnext{Identité remarquable.}
    (x + 5)^2 = 0
        \explnext{$P^2 = 0$ si et 
                  seulement si $P = 0$.}
    x = -5
\end{stepcalc}
\end{latexex}


% ---------------------- %


\newparaexample{Juste utiliser des symboles}

Si l'argument obligatoire de la macro \macro{explnext} est vide alors seul le symbole est affiché \emph{(ne pas oublier les accolades vides)}. Voici un court exemple de ceci.

\begin{latexex}
\begin{stepcalc}[ope = \viff]
    a^2 = b^2
        \explnext{}
    a = \pm b
\end{stepcalc}
\end{latexex}


% ---------------------- %


\newparaexample{Utiliser un autre symbole localement}

La macro \macro{explnext} possède un argument optionnel qui utilise par défaut celui de l'environment. En utilisant cette option, on choisit alors localement le symbole à employer. Voici un exemple d'utilisation complètement farfelu bien que correct.

\begin{latexex}
\begin{stepcalc}[ope = \viff]
    0 \leq a < b
        \explnext[\vimplies]%
                 {Croissance de $x^2$ 
                  sur $R_{+}$.}
    a^2 < b^2
        \explnext{}
    a^2 - b^2 < 0
        \explnext{Identité remarquable.}
    (a - b)(a + b) < 0
        \explnext[\vimplies]{}
    a \neq b
\end{stepcalc}
\end{latexex}


% ---------------------- %


\newparaexample{Choisir la mise en forme des explications}

Pour la mise en forme des explications à double sens, la macro \macro{explnext} fait appel à la macro \macro{expltxt}.
Par défaut, le package utilise la définition suivante.

\begin{latexex-alone}
\newcommand\expltxt[1]{%
    \text{\color{blue}\footnotesize \{\,{\itshape #1}\,\} }%
}
\end{latexex-alone}


Pour la mise en forme des explications à sens unique, la macro \macro{explnext*} fait appel aux macros \macro{expltxtup}, \macro{expltxtdown} et \macro{expltxtupdown}.
Par défaut le package utilise les définitions suivantes.

\begin{latexex-alone}
\newcommand\expltxtup[1]{%
    $\uparrow$ #1 $\uparrow$%
}

\newcommand\expltxtdown[1]{%
    $\downarrow$ #1 $\downarrow$%
}

\newcommand\expltxtupdown[2]{{%
    \displaystyle\footnotesize\color{blue}%
    \left\{\,%
        \genfrac{}{}{0pt}{}{%
            \text{\itshape\expltxtdown{\samesizeas{#1}{#2}}}%
        }{%
            \text{\itshape\expltxtup{\samesizeas{#2}{#1}}}%
        }%
    \,\right\}%
}}
\end{latexex-alone}


Nous allons expliquer comment obtenir l'affreux exemple ci-dessous montrant que l'on peut adapter si besoin la mise en forme.

% ==================== %

\bgroup

\newcommand\myexpltxt[2]{%
    \text{\color{#1} \footnotesize \itshape \bfseries #2}%
}

\renewcommand\expltxt[1]{%
    \myexpltxt{gray}{$\Downarrow$ #1 $\Uparrow$}%
}

\renewcommand\expltxtup[1]{%
    \myexpltxt{orange}{$\Uparrow$ #1 $\Uparrow$}%
}

\renewcommand\expltxtdown[1]{%
    \myexpltxt{red}{$\Downarrow$ #1 $\Downarrow$}%
}

\renewcommand\expltxtupdown[2]{%
    \displaystyle\color{blue!20!black!30!green}\genfrac{\langle}{\rangle}{1pt}{}{%
        \expltxtdown{\samesizeas{#1}{#2}}%
    }{%
        \expltxtup{\samesizeas{#2}{#1}}%
    }%
}


\begin{latexex}
\begin{stepcalc}
    (a + b) (a + b)
        \explnext{Se souvenir de $P\cdot P = P^2$.}%
    (a + b)^2
        \explnext*{Id. Rm. - Dév.}%
                  {Id. Rm. - Facto.}
    a^2 + 2 a b + b^2
\end{stepcalc}
\end{latexex}

\egroup


La mise en forme a été obtenue en utilisant le code \LaTeX{} suivant où la macro \macro{samesizeas\{\#1\}\{\#2\}} rend le texte \verb+#1+ aussi large que \verb+#2+ en ajoutant des espaces supplémentaires tout en centrant le résultat final si besoin  \emph{(ne pas oublier de passer en mode texte via \macro{text})}.

\begin{latexex-alone}
\newcommand\myexpltxt[2]{%
    \text{\color{#1} \footnotesize \itshape \bfseries #2}%
}

\renewcommand\expltxt[1]{%
    \myexpltxt{gray}{$\Downarrow$ #1 $\Uparrow$}%
}

\renewcommand\expltxtup[1]{%
    \myexpltxt{orange}{$\Uparrow$ #1 $\Uparrow$}%
}

\renewcommand\expltxtdown[1]{%
    \myexpltxt{red}{$\Downarrow$ #1 $\Downarrow$}%
}

\renewcommand\expltxtupdown[2]{%
    \displaystyle\color{blue!20!black!30!green}%
    \genfrac{\langle}{\rangle}{1pt}{}{%
        \expltxtdown{\samesizeas{#1}{#2}}%
    }{%
        \expltxtup{\samesizeas{#2}{#1}}%
    }%
}
\end{latexex-alone}


% ==================== %


\end{document}
