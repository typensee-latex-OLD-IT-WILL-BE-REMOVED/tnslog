\documentclass[12pt,a4paper]{article}

\makeatletter
    \usepackage[utf8]{inputenc}
\usepackage[T1]{fontenc}
\usepackage{ucs}

\usepackage[french]{babel,varioref}

\usepackage[top=2cm, bottom=2cm, left=1.5cm, right=1.5cm]{geometry}
\usepackage{enumitem}

\usepackage{pgffor}

\usepackage{multicol}

\usepackage{makecell}

\usepackage{color}
\usepackage{hyperref}
\hypersetup{
    colorlinks,
    citecolor=black,
    filecolor=black,
    linkcolor=black,
    urlcolor=black
}

\usepackage{amsthm}

\usepackage{tcolorbox}
\tcbuselibrary{listingsutf8}

\usepackage{ifplatform}

\usepackage{ifthen}

\usepackage{macroenvsign}


% Sections numbering

%\renewcommand\thechapter{\Alph{chapter}.}
\renewcommand\thesection{\Roman{section}.}
\renewcommand\thesubsection{\arabic{subsection}.}
\renewcommand\thesubsubsection{\roman{subsubsection}.}



% MISC

\newtcblisting{latexex}{%
	sharp corners,%
	left=1mm, right=1mm,%
	bottom=1mm, top=1mm,%
	colupper=red!75!blue,%
	listing side text
}

\newtcbinputlisting{\inputlatexex}[2][]{%
	listing file={#2},%
	sharp corners,%
	left=1mm, right=1mm,%
	bottom=1mm, top=1mm,%
	colupper=red!75!blue,%
	listing side text
}


\newtcblisting{latexex-flat}{%
	sharp corners,%
	left=1mm, right=1mm,%
	bottom=1mm, top=1mm,%
	colupper=red!75!blue,%
}

\newtcbinputlisting{\inputlatexexflat}[2][]{%
	listing file={#2},%
	sharp corners,%
	left=1mm, right=1mm,%
	bottom=1mm, top=1mm,%
	colupper=red!75!blue,%
}


\newtcblisting{latexex-alone}{%
	sharp corners,%
	left=1mm, right=1mm,%
	bottom=1mm, top=1mm,%
	colupper=red!75!blue,%
	listing only
}

\newtcbinputlisting{\inputlatexexalone}[2][]{%
	listing file={#2},%
	sharp corners,%
	left=1mm, right=1mm,%
	bottom=1mm, top=1mm,%
	colupper=red!75!blue,%
	listing only
}


\newcommand\inputlatexexcodeafter[1]{%
	\begin{center}
		\input{#1}
	\end{center}

	\vspace{-.5em}
	
	Le rendu précédent a été obtenu via le code suivant.
	
	\inputlatexexalone{#1}
}


\newcommand\inputlatexexcodebefore[1]{%
	\inputlatexexalone{#1}
	\vspace{-.75em}
	\begin{center}
		\textit{\footnotesize Rendu du code précédent}
		
		\medskip
		
		\input{#1}
	\end{center}
}


\newcommand\env[1]{\texttt{#1}}
\newcommand\macro[1]{\env{\textbackslash{}#1}}



\setlength{\parindent}{0cm}
\setlist{noitemsep}

\theoremstyle{definition}
\newtheorem*{remark}{Remarque}

\usepackage[raggedright]{titlesec}

\titleformat{\paragraph}[hang]{\normalfont\normalsize\bfseries}{\theparagraph}{1em}{}
\titlespacing*{\paragraph}{0pt}{3.25ex plus 1ex minus .2ex}{0.5em}


\newcommand\separation{
	\medskip
	\hfill\rule{0.5\textwidth}{0.75pt}\hfill
	\medskip
}


\newcommand\extraspace{
	\vspace{0.25em}
}


\newcommand\whyprefix[2]{%
	\textbf{\prefix{#1}}-#2%
}

\newcommand\mwhyprefix[2]{%
	\texttt{#1 = #1-#2}%
}

\newcommand\prefix[1]{%
	\texttt{#1}%
}


\newcommand\inenglish{\@ifstar{\@inenglish@star}{\@inenglish@no@star}}

\newcommand\@inenglish@star[1]{%
	\emph{\og #1 \fg}%
}

\newcommand\@inenglish@no@star[1]{%
	\@inenglish@star{#1} en anglais%
}


\newcommand\ascii{\texttt{ASCII}}


% Example
\newcounter{paraexample}[subsubsection]

\newcommand\@newexample@abstract[2]{%
	\paragraph{%
		#1%
		\if\relax\detokenize{#2}\relax\else {} -- #2\fi%
	}%
}



\newcommand\newparaexample{\@ifstar{\@newparaexample@star}{\@newparaexample@no@star}}

\newcommand\@newparaexample@no@star[1]{%
	\refstepcounter{paraexample}%
	\@newexample@abstract{Exemple \theparaexample}{#1}%
}

\newcommand\@newparaexample@star[1]{%
	\@newexample@abstract{Exemple}{#1}%
}


% Change log
\newcommand\topic{\@ifstar{\@topic@star}{\@topic@no@star}}

\newcommand\@topic@no@star[1]{%
	\textbf{\textsc{#1}.}%
}

\newcommand\@topic@star[1]{%
	\textbf{\textsc{#1} :}%
}



    \usepackage{01-stepcalc}
\makeatother


\begin{document}

%\section{Logique et fondements}

\section{Fiches techniques}

\subsection{Détailler un raisonnement simple} 

\IDenv[o]{stepcalc}{1}

\IDoption{} la valeur utilise une syntaxe de type clé-valeur. Voici les différentes clés disponibles.

\begin{enumerate}
	\item \verb+ope+ sert à définir l'opérateur utilisé dans tout l'environnement qui sera rédigé en mode mathématique. 
	      La valeur par défaut est \verb+{=}+ \emph{(et non juste \texttt{=} attention)}.

	\item \verb+style+ sert à définir le style de mise en forme. Voici les différentes valeurs possibles.
	      \begin{enumerate}
	      		\item \prefix{u}, la valeur par défaut, est pour \whyprefix{u}{niversity}.

	      		\item \prefix{ar} est pour \whyprefix{ar}{row}.

	      		\item \prefix{ar*} est similaire à \prefix{ar}  mais avec l'opérateur dans la marge.

	      		\item \prefix{sar} est pour \whyprefix{s}{hort} \whyprefix{ar}{row}.
	      \end{enumerate}

	\item \verb+com+ permet de demander l'alignement ou non des commentaires non étoilés entre eux.
	      \begin{enumerate}
	      		\item \prefix{nal}, la valeur par défaut, est pour \whyprefix{n}{ot} \whyprefix{al}{igned}.

	      		\item \prefix{al} est pour \whyprefix{al}{igned}.

	      \end{enumerate}
\end{enumerate}


\separation


\IDmacro{explnext}{1}{1}  \hfill \mwhyprefix{expl}{ain}

\IDoption{} le symbole à utiliser pour une explication, la valeur par défaut étant celle du symbole de l'environnement \env{stepcalc} où \macro{explnext} est utilisé.

\IDarg{} le texte de l'explication qui peut être vide si aucune explication n'est à afficher.

\medskip

\medskip

\emph{\textbf{ATTENTION !} La macro \macro{explnext} est à utiliser sans argument ni option au tout début du contenu de l'environnement \env{stepcalc} en cas d'utilisation du style \texttt{sar}.}


\separation


\IDmacro{explnext*}{1}{2}

\IDoption{} le symbole à utiliser pour une explication, la valeur par défaut étant celle du symbole de l'environnement \env{stepcalc} où \macro{explnext} est utilisé.

\IDarg{1} le texte de l'explication pour la 1\iere{} ligne.
          Ce texte peut être vide \emph{(voir l'environnement \env{astepcalc} pour la raison de ceci)}.

\IDarg{2} le texte de l'explication pour la 2\ieme{} ligne.
          Ce texte peut être vide \emph{(voir l'environnement \env{astepcalc} pour la raison de ceci)}.


\separation


\IDmacro[a]{comthis }{1}  \hfill \mwhyprefix{com}{ment}

\IDmacro[a]{comthis*}{1}


\IDarg{} le texte d'un court commentaire.


% ---------------------- %


\subsection{Détailler un raisonnement simple -- Mise en forme du texte}

Les macros suivantes sont juste utilisées par l'environnement \env{stepcalc}.


\separation


\IDmacro[n]{expltxtspacein}


\separation


\IDmacro[a]{expltxt}{1}

\IDarg{} le texte de l'explication que l'on veut mettre en forme.


\separation


\IDmacro[a]{expltxtdown}{1}

\IDarg{} le texte de l'explication du haut vers le bas que l'on veut mettre en forme.


\separation


\IDmacro[a]{expltxtup}{1}

\IDarg{} le texte de l'explication du bas vers le haut que l'on veut mettre en forme.


\separation


\IDmacro[a]{expltxtupdown}{2}

\IDarg{1} le texte de l'explication du haut vers le bas que l'on veut mettre en forme.

\IDarg{1} le texte de l'explication du bas vers le haut que l'on veut mettre en forme.


\separation


\IDmacro[a]{explcom}{1}

\IDarg{} le texte d'un court commentaire.


\end{document}