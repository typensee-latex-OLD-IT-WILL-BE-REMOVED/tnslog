\documentclass[12pt,a4paper]{article}

\makeatletter
    \usepackage[utf8]{inputenc}
\usepackage[T1]{fontenc}
\usepackage{ucs}

\usepackage[french]{babel,varioref}

\usepackage[top=2cm, bottom=2cm, left=1.5cm, right=1.5cm]{geometry}
\usepackage{enumitem}

\usepackage{pgffor}

\usepackage{multicol}

\usepackage{makecell}

\usepackage{color}
\usepackage{hyperref}
\hypersetup{
    colorlinks,
    citecolor=black,
    filecolor=black,
    linkcolor=black,
    urlcolor=black
}

\usepackage{amsthm}

\usepackage{tcolorbox}
\tcbuselibrary{listingsutf8}

\usepackage{ifplatform}

\usepackage{ifthen}

\usepackage{macroenvsign}


% Sections numbering

%\renewcommand\thechapter{\Alph{chapter}.}
\renewcommand\thesection{\Roman{section}.}
\renewcommand\thesubsection{\arabic{subsection}.}
\renewcommand\thesubsubsection{\roman{subsubsection}.}



% MISC

\newtcblisting{latexex}{%
	sharp corners,%
	left=1mm, right=1mm,%
	bottom=1mm, top=1mm,%
	colupper=red!75!blue,%
	listing side text
}

\newtcbinputlisting{\inputlatexex}[2][]{%
	listing file={#2},%
	sharp corners,%
	left=1mm, right=1mm,%
	bottom=1mm, top=1mm,%
	colupper=red!75!blue,%
	listing side text
}


\newtcblisting{latexex-flat}{%
	sharp corners,%
	left=1mm, right=1mm,%
	bottom=1mm, top=1mm,%
	colupper=red!75!blue,%
}

\newtcbinputlisting{\inputlatexexflat}[2][]{%
	listing file={#2},%
	sharp corners,%
	left=1mm, right=1mm,%
	bottom=1mm, top=1mm,%
	colupper=red!75!blue,%
}


\newtcblisting{latexex-alone}{%
	sharp corners,%
	left=1mm, right=1mm,%
	bottom=1mm, top=1mm,%
	colupper=red!75!blue,%
	listing only
}

\newtcbinputlisting{\inputlatexexalone}[2][]{%
	listing file={#2},%
	sharp corners,%
	left=1mm, right=1mm,%
	bottom=1mm, top=1mm,%
	colupper=red!75!blue,%
	listing only
}


\newcommand\inputlatexexcodeafter[1]{%
	\begin{center}
		\input{#1}
	\end{center}

	\vspace{-.5em}
	
	Le rendu précédent a été obtenu via le code suivant.
	
	\inputlatexexalone{#1}
}


\newcommand\inputlatexexcodebefore[1]{%
	\inputlatexexalone{#1}
	\vspace{-.75em}
	\begin{center}
		\textit{\footnotesize Rendu du code précédent}
		
		\medskip
		
		\input{#1}
	\end{center}
}


\newcommand\env[1]{\texttt{#1}}
\newcommand\macro[1]{\env{\textbackslash{}#1}}



\setlength{\parindent}{0cm}
\setlist{noitemsep}

\theoremstyle{definition}
\newtheorem*{remark}{Remarque}

\usepackage[raggedright]{titlesec}

\titleformat{\paragraph}[hang]{\normalfont\normalsize\bfseries}{\theparagraph}{1em}{}
\titlespacing*{\paragraph}{0pt}{3.25ex plus 1ex minus .2ex}{0.5em}


\newcommand\separation{
	\medskip
	\hfill\rule{0.5\textwidth}{0.75pt}\hfill
	\medskip
}


\newcommand\extraspace{
	\vspace{0.25em}
}


\newcommand\whyprefix[2]{%
	\textbf{\prefix{#1}}-#2%
}

\newcommand\mwhyprefix[2]{%
	\texttt{#1 = #1-#2}%
}

\newcommand\prefix[1]{%
	\texttt{#1}%
}


\newcommand\inenglish{\@ifstar{\@inenglish@star}{\@inenglish@no@star}}

\newcommand\@inenglish@star[1]{%
	\emph{\og #1 \fg}%
}

\newcommand\@inenglish@no@star[1]{%
	\@inenglish@star{#1} en anglais%
}


\newcommand\ascii{\texttt{ASCII}}


% Example
\newcounter{paraexample}[subsubsection]

\newcommand\@newexample@abstract[2]{%
	\paragraph{%
		#1%
		\if\relax\detokenize{#2}\relax\else {} -- #2\fi%
	}%
}



\newcommand\newparaexample{\@ifstar{\@newparaexample@star}{\@newparaexample@no@star}}

\newcommand\@newparaexample@no@star[1]{%
	\refstepcounter{paraexample}%
	\@newexample@abstract{Exemple \theparaexample}{#1}%
}

\newcommand\@newparaexample@star[1]{%
	\@newexample@abstract{Exemple}{#1}%
}


% Change log
\newcommand\topic{\@ifstar{\@topic@star}{\@topic@no@star}}

\newcommand\@topic@no@star[1]{%
	\textbf{\textsc{#1}.}%
}

\newcommand\@topic@star[1]{%
	\textbf{\textsc{#1} :}%
}



    \usepackage{01-calc-explained}
\makeatother


\begin{document}

%\section{Détailler un raisonnement simple}

\subsection{Version pour les collégiens} \label{tnslog-explain-proof-for-youngs}

L'environnement \env{explain} avec l'option \verb+style = ar+
\footnote{
    Cet environnement utilise aussi le package \texttt{witharrows}.
}
utilise des flèches pour indiquer les explications
\emph{(\prefix{ar} est pour \whyprefix{ar}{row} soit \inenglish{flèche})}.
Dans ce cas d'utilisation, la macro \macro{explnext*} permet d'avoir une flèche unidirectionnelle, vers le haut ou le bas au choix, ou bien d'écrire deux indications dont l'une est montante et l'autre descendante.

\medskip

Il existe aussi l'option \verb+style = sar+ lorsque la toute 1\iere{} étape n'est pas expliquée
\emph{(\prefix{s} est pour \whyprefix{s}{hort} soit \inenglish{court})}.
Attention car forcément ceci nécessite au tout début de l'environnement l'usage de la macro \macro{explnext} sans aucun contenu !


% ---------------------- %


\newparaexample{Des flèches à double sens}

\begin{latexex}
\begin{explain}[style = ar]
    (a + b)^2
        \explnext{Identité remarquable}
    a^2 + 2 a b + b^2
        \explnext{}
    a^2 + b^2 + 2 a b
\end{explain}
\end{latexex}


% ---------------------- %


\newparaexample{Des flèches unidirectionnelles}

Ce qui suit est juste là comme démo. car les explications y sont un peu farfelues.

\begin{latexex-flat}
\begin{explain}[style = ar]
    (a + b)^2
        \explnext*{Via $P^2 = P \cdot P$.}
                  {Via $P \cdot P = P^2$.}
    (a + b) (a + b)
        \explnext*{Double développement.}%
                  {Double factorisation (pas simple).}
    a^2 + a b + b a + b^2
        \explnext*{Commutativité du produit.}%
                  {}
    a^2 + 2 a b + b^2
        \explnext*{}%
                  {Commutativité de l'addition.}
    a^2 + b^2 + 2 a b
\end{explain}
\end{latexex-flat}


% ---------------------- %


\newparaexample{Ne pas expliquer le tout début}

L'environnement étoilé \env{explain} avec l'option \verb+style = sar+ débute différemment la mise en forme.
Bien entendu ici le tout premier \explnext{} doit avoir un argument vide !

\begin{latexex-flat}
\begin{explain}[style = sar]
    (a + b) (a + b)
        \explnext{}
    (a + b)^2
        \explnext{Identité remarquable.}
    a^2 + b^2 + 2 a b
\end{explain}
\end{latexex-flat}


% ---------------------- %


\newparaexample{Choisir son symbole}

Voici comment faire où l'implication finale est juste là pour la démonstration \emph{(on notera une petite bidouille un peu sale à faire pour avoir un alignement à peu près correct)}.

\begin{latexex}
\begin{explain}[style = ar, ope = \iff]
    a^2 + 2 a b + b^2 = 0
        \explnext{}
    (a + b)^2 = 0
        \explnext[\:\implies]%
                 {$P^2 = 0$ ssi $P = 0$.}
    a + b = 0
\end{explain}
\end{latexex}


Avec la version courte, on obtient ce qui suit.

\begin{latexex-flat}
\begin{explain}[style = sar, ope = \iff]
    a^2 + 2 a b + b^2 = 0
        \explnext{}
    (a + b)^2 = 0
        \explnext[\:\implies]%
                 {$P^2 = 0$ ssi $P = 0$.}
    a + b = 0
\end{explain}
\end{latexex-flat}

\end{document}
